Lisp and Scheme have demonstrated the power of macros to enable programmers to evolve and craft languages. Mozilla Sweet.JS provides a way for developers to enrich their JavaScript code by adding new syntax to the language through the use of macros. Sweet.JS provides the possibility to define hygienic macros inspired by Scheme.

\textcolor{white}{``}In this paper, I present the implementation of a ``syntax parameter'' feature for the Sweet.JS library. A syntax parameter is a mechanism for rebinding a macro definition within the dynamic context of a macro expansion. Some time hygienic macro bindings are insufficient such as with ``anaphoric conditionals" where the value of the tested expression is available as an {\it it} binding. Syntax parameter are a outstanding instrument for resolving the setback of macro that demand to attach a recognized name. With syntax parameters, instead of introducing the binding unhygienically each time, we instead create one binding for the keyword, which we can then adjust later when we want the keyword to have a different meaning. As no new bindings are introduced hygiene is preserved. In my implementation I define ``syntaxparam,'' which defines and binds the syntax parameter part of the compiler; ``syntaxLocalValue,'' which pulls the syntax parameter definition in the defined scope and ``replaceSyntaxParamTransform,'' which expands the syntax parameter macro definition defined within the macro body using ``syntaxLocalValue''.\textcolor{white}{"}